% Draft of paper for Zone based traffic-sensitive routing
% The aim here is to show an increase in scalibility with a preservation of traffic reduction (show that the zone system will work with any model)

\documentclass[conference]{IEEEtran}
\usepackage{mathtools}

\usepackage{algorithm}% http://ctan.org/pkg/algorithms
\usepackage{algpseudocode}% http://ctan.org/pkg/algorithmicx

\usepackage{graphicx}
\graphicspath{ {images/} }

\hyphenation{op-tical net-works semi-conduc-tor}

\begin{document}
\raggedbottom

\title{Working Title: Z-BAR: A Zone-Based traffic Assignment Algorithm for scalable congestion Reduction}

\author{\IEEEauthorblockN{Taylor Cox}
\IEEEauthorblockA{Department of Computer Science\\
University of Manitoba\\
Winnipeg, Manitoba\\
Email: coxt3@myumanitoba.ca}}

\maketitle

\begin{abstract}


\end{abstract}

\begin{IEEEkeywords}
Zones, Traffic Assignment, Congestion Reduction, Scalibility
\end{IEEEkeywords}

\IEEEpeerreviewmaketitle

% outline from meeting has section breakdown:
\section{Introduction}

Intelligent Transportation Systems (ITS) aim to improve urban transit and travel experiences with the use of modern technology. This research area contributes to the development of smart cities. One criterion for the success of a smart city is the optimization of traffic flow. Traffic assignment optimization is an emerging subarea of ITS. Recent work in traffic assignment optimization literature aims to positively influence driver behavior without requiring any individual drivers to embark on less optimal routes. In this manner, a balance is struck between driver selfishness and system optimum.

Traffic assignment optimization is a research area with significant social and ecological impact if left uninvestigated. The optimization of traffic routing in urban centers is crucial to addressing grand-scale socioecological issues such as climate change. In a natural system, drivers exhibit selfish behavior when traveling in urban centers. This creates significant congestion in key areas, such as a city's downtown core. Such congestion leads to runaway emissions of carbon-based pollutants, and imposes significant impact onto day-to-day productivity.

The goal of this paper is to investigate strategies for scaling existing traffic assignment systems to very large datasets. This includes both artificial and real-world traffic scenarios. This paper aims to apply existing congestion reduction strategies against large, realistic inputs. In this work, cities are treated as discrete-time, multi-agent systems with no central control. This paper examines a specific strategy for the scalability of traffic assignment optimization. This strategy is one based on partial centralization. We propose that traffic networks be partitioned into zones, where the path between all pairs of intersections within one zone is stored in advance and evaluated at the zone-level. This reduces the time taken for individual vehicles to evaluate an optimal path. This zone-based approach is implemented in an algorithm dubbed Z-BAR, which is examined in detail later in this work.

An ACO-based traffic assignment algorithm inspired by [*] is implemented in this work, and used as a template for illustrating the scalability Z-BAR provides. Since Z-BAR's purpose is scalability as opposed to traffic assignment itself, any decentralized traffic assignment model will be compatible with Z-BAR. With scaling provided by Z-BAR, the traffic assignment algorithm is executed against mutliple artificial and real-world scenarios of increasing network and population size. To the knowledge of the authors, this is the first work which specifically addresses scalability of traffic assignment optimization algorithms, and does so using the concept of zones.

The remainder of the paper is organized as follows: Section 2 provides an overview of relevant work; Section 2 describes the paper's algorithm Z-BAR; Section 4 describes the simulation and experimental environment; Section 5 provides the experimental results against select large datasets. Finally, Section 6 presents the conclusions of this paper, accompanied by a discussion of future work.

% objective: traffic reduction scaled %

\section{Related Work} % existing lit and particularly IACO - and its issues. Also need HOPNET for foundation
% related work really splits into two parts.

% go through the relevant papers. IACO DTPOS in particular
The traffic assignment problem has been studied thoroughly in the literature, with foundations dating as far back as [*] and [*]. In recent works, Ant Colony Optimization (ACO) has proved to be successful in discovering new techniques for traffic assignment optimization. Originally proposed in [*], ACO is implemented by a set of ants acting upon a graph G = (V, E). Ants deposit pheromone on the routes they traverse, signalling the strength of their path to future ants. The ACO technique has been used in the traffic assignment problem by representing vehicles as ants. Vehicles leave a pheromone signal on the roads they traverse, permitting a characterization of vehicle density on an edge E. This vehicle-as-ant approach has been explored in [*] and [*]. The results in both works show that an ACO system reduces overall travel and wait times, but does not require any vehicles to embark on a suboptimal path in order to achieve a global optimum. Some vehicles may take physically longer routes to their destination, but none will will increase their travel time. Despite their success, these works were not applied to datasets of realistic scale. Both [*] and [*] were tested on relatively small artificial and real-world traffic networks, with up to 10,000 vehicles in a single simulation. Addtional work must be done to ensure the scalability of any traffic assignment technique.

Outside of traffic optimization, there is work done in the literature which covers the problem of scalability in graphs. Much of this work is focused on optimizing routing algorithms in large networks. In the study of networking, routing algorithms are divided into proactive, reactive and hybrid algorithms. In a proactive routing environment, nodes continuously broadcast their routing tables to their neighbours. This guarantees consistency of optimal routes but comes with a high cost of continued route maintenance. Alternatively, reactive routing involves nodes evaluating optimal paths on demand. While routes do not need to be maintained in a reactive protocol, the cost of evaluating new routes can be high in sufficiently large networks. Hybrid routing protocols are those which combine aspects from both proactive and reactive protocols. Some hybrid routing protocols are able to use the advantages of both proactive and reactive protocols without introducing any new disadvantages. One such hybrid protocol is the Zone Routing Protocol (ZRP). The ZRP has was first introduced in [*] and has since been used in a variety of applications. In a zone routing environment, the network is divided into zones, where nodes may belong to one or more overlapping zones. Within each zone, the path between each pair of vertices v1 and v2 is always known in advance. Intrazone paths are updated frequently, creating a proactive routing environment within each zone. When a multi-zone route is required, the local routes stored within each zone is joined together to form a path which traverses multiple zones. This reduces the total amount of message delivery by resorting to reactive routing only when necessary. Applications of the zone routing protocol has been studied widely in [*] and [*], showing the usefulness of zones in mobile and vehicular ad-hoc networks (MANET/VANET). 

\begin{figure}[h]
\caption{Example of multi-zone routing}
\centering
\includegraphics[scale=0.5]{zones_example}
\end{figure}

In the literature, zone-based routing has been successful in a wide array of applications. It is anticipated that a zone-routing strategy will be successful in improving the scalibility of existing traffic assignment algorithms. Storing optimal (least congested) intrazone paths and evaluated interzone paths on demand is a more scalable alternative to evalauting whole-network paths on every vehicle traversal. To the knowledge of the author, the zone routing protocol has not been applied to the traffic assignment optimization problem.

\section{Z-BAR} % how we will tackle the issues, give the algorithm

% the approach and the algorithm. Effectively standard multi-zone routing. Intra: Storage. Inter: React. Dijkstra's for shortest path. Parallelism and async.
\section{Simulation}

\section{Experimental Results} % exec over large scale data (Continue getting script to run)

\section{Conclusion} % what did we find?

\section{Future Work} % next...

\section*{Acknowledgment}

\begin{thebibliography}{1}
\bibitem{}
\end{thebibliography}

\end{document}


