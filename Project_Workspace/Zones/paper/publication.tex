% Draft of paper for Zone based traffic-sensitive routing
% The aim here is to show an increase in scalibility with a preservation of traffic reduction (show that the zone system will work with any model)

\documentclass[conference]{IEEEtran}
\usepackage{mathtools}

\usepackage{algorithm}% http://ctan.org/pkg/algorithms
\usepackage{algpseudocode}% http://ctan.org/pkg/algorithmicx

\hyphenation{op-tical net-works semi-conduc-tor}

\begin{document}
\raggedbottom

\title{Working Title: Z-BAR: A Zone-Based traffic Assignment Algorithm for scalable congestion Reduction}

\author{\IEEEauthorblockN{Taylor Cox}
\IEEEauthorblockA{Department of Computer Science\\
University of Manitoba\\
Winnipeg, Manitoba\\
Email: coxt3@myumanitoba.ca}}

\maketitle

\begin{abstract}


\end{abstract}

\begin{IEEEkeywords}
Zones, Traffic Assignment, Congestion Reduction, Scalibility
\end{IEEEkeywords}

\IEEEpeerreviewmaketitle

% outline from meeting has section breakdown:
\section{Introduction}

\section{Related Work} % existing lit and particularly IACO - and its issues. Also need HOPNET for foundation

\section{Z-BAR} % how we will tackle the issues, give the algorithm

\section{Experimental Results} % exec over large scale data

\section{Conclusion} % what did we find?

\section{Future Work} % next...

\section*{Acknowledgment}

\begin{thebibliography}{1}
\bibitem{}
\end{thebibliography}

\end{document}


