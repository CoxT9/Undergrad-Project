% Draft of paper for Zone based traffic-sensitive routing
% The aim here is to show an increase in scalibility with a preservation of traffic reduction (show that the zone system will work with any model)

\documentclass[conference]{IEEEtran}
\usepackage{mathtools}

\usepackage{algorithm}% http://ctan.org/pkg/algorithms
\usepackage{algpseudocode}% http://ctan.org/pkg/algorithmicx

\hyphenation{op-tical net-works semi-conduc-tor}

\begin{document}
\raggedbottom

\title{Working Title: Z-BAR: A Zone-Based traffic Assignment Algorithm for scalable congestion Reduction}

\author{\IEEEauthorblockN{Taylor Cox}
\IEEEauthorblockA{Department of Computer Science\\
University of Manitoba\\
Winnipeg, Manitoba\\
Email: coxt3@myumanitoba.ca}}

\maketitle

\begin{abstract}


\end{abstract}

\begin{IEEEkeywords}
Zones, Traffic Assignment, Congestion Reduction, Scalibility
\end{IEEEkeywords}

\IEEEpeerreviewmaketitle

% outline from meeting has section breakdown:
\section{Introduction}

Intelligent Transportation Systems (ITS) aim to improve urban transit and travel experiences with the use of modern technology. This research area contributes to the development of smart cities. One criterion for the success of a smart city is the optimization of traffic flow. Traffic assignment optimization is an emerging subarea of ITS. Recent work in traffic assignment optimization literature aims to positively influence driver behavior without requiring any individual drivers to embark on less optimal routes. In this manner, a balance is struck between driver selfishness and system optimum. \\

Traffic assignment optimization is a research area with significant social and ecological impact if left uninvestigated. The optimization of traffic routing in urban centers is crucial to addressing grand-scale socioecological issues such as climate change. In a natural system, drivers exhibit selfish behavior when traveling in urban centers. This creates significant congestion in key areas, such as a city's downtown core. Such congestion leads to runaway emissions of carbon-based pollutants, and imposes significant impact onto day-to-day productivity. \\

The goal of this paper is to investigate strategies for scaling existing traffic assignment systems to very large datasets. This includes both artificial and real-world traffic scenarios. This paper aims to apply existing congestion reduction strategies against large, realistic inputs. In this work, cities are treated as discrete-time, multi-agent systems with no central control. This paper examines a specific strategy for the scalability of traffic assignment optimization. This strategy is one based on partial centralization. We propose that traffic networks be partitioned into zones, where the path between all pairs of intersections within one zone is stored in advance and evaluated at the zone-level. This reduces the time taken for individual vehicles to evaluate an optimal path. This zone-based approach is implemented in an algorithm dubbed Z-BAR, which is examined in detail later in this work. \\

An ACO-based traffic assignment algorithm inspired by [*] is implemented in this work, and used as a template for illustrating the scalability Z-BAR provides. Since Z-BAR's purpose is scalability as opposed to traffic assignment itself, any decentralized traffic assignment model will be compatible with Z-BAR. With scaling provided by Z-BAR, the traffic assignment algorithm is executed against mutliple artificial and real-world scenarios of increasing network and population size. To the knowledge of the authors, this is the first work which specifically addresses scalability of traffic assignment optimization algorithms, and does so using the concept of zones. \\

The remainder of the paper is organized as follows: Section 2 provides an overview of relevant work; Section 2 describes the paper's algorithm Z-BAR; Section 4 describes the simulation and experimental environment; Section 5 provides the experimental results against select large datasets. Finally, Section 6 presents the conclusions of this paper, accompanied by a discussion of future work.

% objective: traffic reduction scaled %

\section{Related Work} % existing lit and particularly IACO - and its issues. Also need HOPNET for foundation
% related work really splits into two parts.

In the recent literature, traffic assignment optimizations have been found which reduce overall travel and wait times for all vehicles in a given system. While some vehicles may take physically longer routes, no vehicles are forced to spend more time on the road in order to advance a greater good. A number of different techniques for traffic assignment optimization and traffic congestion reduction have been explored in the literature. \\

% go through the relevant papers. IACO DTPOS in particular

Seperately, the recent literature also includes approaches for scaling the effectiveness of algorithms which cover large graphs. Much of the work in this section is derived from the study of networking. Proactive (routes stored in advance) and reactive (routes evaluated on-demand) routing strategies are both examined. Additionally, hybird strategies are also examined, where proactive and reactive routing is used in the same system. The hybrid routing strategy of note is the zone routing protocol (ZRP)

% dig into zone approaches. HOPNET and MAZACORNET.

% propose a combination of the two strategies: zones for a new type of graph/network

\section{Z-BAR} % how we will tackle the issues, give the algorithm

\section{Simulation}

\section{Experimental Results} % exec over large scale data

\section{Conclusion} % what did we find?

\section{Future Work} % next...

\section*{Acknowledgment}

\begin{thebibliography}{1}
\bibitem{}
\end{thebibliography}

\end{document}


